\paragraph{Notation.}
% Note: Recall that a (total) function maps every element in its domain. 
% Its range are all the values that are mapped to some domain key.
% Its codomain are all the possible values in the universe, including unmapped ones. 
% In contrast, partial function does not map every element in the domain.
We introduce some helpful notation for (authenticated) sets and dictionaries, which are the focus of this part of the thesis.
Let $\primes(\lambda)$ denote the set of all $\lambda$-bit prime numbers.
Let $|S|$ denote the number of elements in a multiset $S$ (e.g., $S=\{1,2,2\}$ and $|S|=3$).
For dictionaries, let $\mathcal{K}$ be the set of all possible keys and $\mathcal{V}$ be the set of all possible values.
($\mathcal{K}$ and $\mathcal{V}$ are application-specific; e.g., in software transparency, a key is the software package name and a value is the hash of a specific version of this software package.)
Let $\mathcal{P}(\mathcal{V})$ denote all possible multisets with elements from $\mathcal{V}$.
Let $K\subset \mathcal{K}$.
Then, a \textit{dictionary} is a function $D : K \rightarrow \mathcal{P}(\mathcal{V})$ that maps a key $k\in K$ to a multiset of values $V\in\mathcal{P}(\mathcal{V})$ (including the empty set).
Let $D(k)$ denote the multiset of values associated with key $k$ in dictionary $D$.
Let $|D|$ denote the number of key-value pairs in the dictionary.
We also refer to this as the \textit{version} of the dictionary.
Appending $(k,v)$ to a version $i$ dictionary updates the multiset $V = D(k)$ of key $k$ to $V' = V \cup \{v\}$ and increments the dictionary version to $i+1$.
Finally, let $\varepsilon$ denote the empty string.

\section{Cryptographic Assumptions}
\label{s:prelim:hidden-order-groups}
% Q: Must these hidden-order groups be cyclic? Is g in Strong RSA assumption a generator?
Some of the results in this thesis (see \cref{s:aas:from-rsa-acc}) are based on cryptosystems which use \textit{hidden-order groups} such as $\ZNstar=\{a \mid \gcd(a,N) = 1\}$ where $N=pq$ is the product of two primes~\cite{rsa}.
Let $\GroupHidOrd \leftarrow \grouphosetup(1^\lambda)$ denote an algorithm for generating such groups, where $\lambda$ is a security parameter.
In addition to classical assumptions, such as the \textit{discrete log (DL) assumption} and the \textit{RSA assumption}, this thesis relies on the \textit{Strong RSA assumption} and the \textit{Adaptive Root assumption}, which we describe below.

\subsection{Strong RSA Assumption}
\label{s:prelim:strong-rsa}

\begin{definition}[Strong RSA Assumption]
\label{def:strong-rsa}
$\forall$ adversaries \Adv running in time $\poly(\lambda)$:
\begin{align*}
    \Pr \left[\begin{array}{c}
        \GroupHidOrd \leftarrow \grouphosetup(1^\lambda), g \in_R \GroupHidOrd,\\
        (u, e) \leftarrow \Adv(1^\lambda, \GroupHidOrd, g):\\
        u^e = g\ \text{and}\ e\ \text{is prime}
    \end{array}
    \right] \le \negl(\lambda)
\end{align*}
\end{definition}

Informally, this assumption says that no computationally-bounded adversary can compute \textit{any} $e$th prime root of a random element $g$.
This is a generalization of the RSA assumption which says that, for a \textit{fixed} $e$, no computationally-bounded adversary can compute an $e$th root of a random $g$.

\subsection{Adaptive Root Assumption}
\label{s:prelim:adaptive-root-assumption}

% See Section 6.1: The Root Finding Problem
This assumption was introduced by Wesolowski~\cite{Wesolowski19}.

\begin{definition}[Adaptive Root Assumption]
\label{def:adaptive-root}
$\forall$ adversaries $\Adv=(\Adv_0,\Adv_1)$ running in time $\poly(\lambda)$:
\begin{align*}
    \Pr \left[\begin{array}{c}
        \GroupHidOrd \leftarrow \grouphosetup(1^\lambda),\\
        (w, \mathsf{state}) \leftarrow \Adv_0(1^\lambda, \GroupHidOrd),\\
        e \in_R \primes(\lambda),\\
        u \leftarrow \Adv_1(1^\lambda, e, \mathsf{state}):\\
        {u^e = w} \wedge {w\ne 1}
    \end{array}
    \right] \le \negl(\lambda)
\end{align*}
\end{definition}

For this assumption to hold, there cannot be any known-order elements in $\GroupHidOrd$.
This means $\GroupHidOrd = \ZNstar$ should \textbf{not} be used directly.
Instead, $\ZNstar \backslash \{\pm 1\}$ should be used, as described in ~\cite{BBF19}.
Finally, note that the Adaptive Root Assumption and the Strong RSA Assumption (see \cref{def:strong-rsa}) are incomparable.
In the former, the adversary picks the base and is asked to compute a specific, randomly-picked root of that base.
In the latter, the adversary is randomly given a base and asked to compute \textit{any} prime root of that base.