% $Log: abstract.tex,v $
% Revision 1.1  93/05/14  14:56:25  starflt
% Initial revision
% 
% Revision 1.1  90/05/04  10:41:01  lwvanels
% Initial revision
% 
%
%% The text of your abstract and nothing else (other than comments) goes here.
%% It will be single-spaced and the rest of the text that is supposed to go on
%% the abstract page will be generated by the abstractpage environment.  This
%% file should be \input (not \include 'd) from cover.tex

Despite 40+ years of amazing progress, cryptography is constantly plagued by two simple problems: keeping secret keys \textit{secret} and making public keys \textit{public}.
For example, public-key encryption is secure only if each user (1) keeps his secret key out of the hands of the adversary and (2) correctly distributes his public key to all other users.
This thesis seeks to address these two fundamental problems.

First, we introduce communication-efficient, fully-untrusted append-only logs, which can be used to correctly distribute public keys. 
Our constructions have logarithmic-sized proofs for the two key operations in append-only logs: looking up public keys and verifying the log remained append-only. 
In contrast, previous logs either have linear-sized proofs or need extra trust assumptions.
%Although our constructions offer small proof sizes, they are not as fast to append to as Merkle-based logs.
Our logs can also be used to secure software distribution and, we hope, to increase transparency in any institution that wants to do so.

Second, we speed up threshold cryptosystems, which protect secret keys by splitting them up across many users.
We introduce threshold signatures, verifiable secret sharing and distributed key generation protocols that can scale to millions of users. 
Our protocols drastically reduce execution time, anywhere from 2\texttimes\xspace to 4500\texttimes, depending on the scale. 
For example, at large scales, we reduce time from tens of hours to tens of seconds.

At the core of most of our contributions lie new techniques for computing evaluation proofs in constant-sized polynomial commitments.
Specifically, we show how to decrease the time to compute $n$ proofs for a degree-bound $n$ polynomial from $O(n^2)$ to $O(n\log{n})$, at the cost of increasing proof size from $O(1)$ to $O(\log{n})$. 
Our techniques could be of independent interest, as they give rise to other cryptographic schemes, such as Vector Commitments (VCs).

