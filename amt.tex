In this section, we improve KZG's $\Theta(nt)$ time for computing $n$ proofs for a degree-bound $t$ polynomial to $\Theta(n\log{t})$ time.

\section{Authenticated Polynomial Multipoint Evaluation Trees (AMTs)}
Our key technique is to commit to the quotients in a polynomial multipoint evaluation (see~\cref{s:prelim:multipoint-eval}), obtaining an \textit{authenticated multipoint evaluation tree} (AMT).
However, our new AMT evaluation proofs are logarithmic-sized, whereas KZG proofs are constant-sized.
Throughout this section, we restrict ourselves to computing AMTs at points $\{1,2,\dots,n\}$ on polynomials of degree $t-1 < n$.
(In \cref{s:amt:arbitrary-points}, we discuss generalizing to any set of points.)
Finally, in \cref{s:amt:proofs}, we show AMT evaluation proofs are secure under $q$-SBDH.
In contrast, KZG proofs are secure under a weaker assumption called $q$-SDH~\cite{BB08}.

\subsection{Computing $n$ Evaluations Proofs in $n\log^2{n}$ time}
\label{s:amt:computing-proofs}
KZG evaluation proofs leverage the \textit{polynomial remainder theorem}: $\forall i\in \Fp, \exists q_i$ of degree $t-1$ such that $\phi(x) = q_i(x)(x-i) + \phi(i)$.
Specifically, a constant-sized KZG proof for $\phi(i)$ is just a commitment to the quotient polynomial $q_i$ (see \cref{s:prelim:polycommit}) and takes $\Theta(t)$ time to compute.
Thus, computing KZG proofs for each $i\in[n]$ takes $\Theta(nt)$ time.
We improve on this by looking at $\phi(x)$ from the lens of a polynomial multipoint evaluation~\cite{vG13ModernCh10}.

For example, consider the multipoint evaluation of $\phi$ at all $i\in[8]$ from \cref{f:multipoint-eval}.
Note that every node in the multipoint evaluation tree stores a quotient and a remainder obtained by dividing the parent node's remainder by its accumulator polynomial (see \cref{s:prelim:multipoint-eval}).
The first key idea is that, for any evaluated point $i\in[8]$, $\phi(x)$ can be expressed as $\phi(i)$ plus a linear combination of quotients and accumulator polynomials along the path to $\phi(i)$'s leaf in the multipoint evaluation tree.

For example, consider $i=1$, which has the left-most path in tree.
Start with the root node in \cref{f:multipoint-eval}, which says:
$$\phi(x)=q_{1,8}(x) (x-1)\dots(x-8) + r_{1,8}(x).$$
Then, expand $r_{1,8}(x)$ by going left in the tree (down towards $\phi(1)$'s leaf), obtaining:
\begin{align*}
\phi(x)& = q_{1,8}(x) (x-1)(x-2)(x-3)(x-4)\cdots(x-8) \\
       & + q_{1,4}(x) (x-1)(x-2)(x-3)(x-4) + r_{1,4}.
\end{align*}
Repeat this process recursively by replacing $r_{1,4}(x)$ and then $r_{1,2}(x)$ to get:
\begin{align*}
\phi(x)& = q_{1,8}(x) (x-1)(x-2)(x-3)(x-4)\cdots(x-8) \\
       & + q_{1,4}(x) (x-1)(x-2)(x-3)(x-4) \\
       & + q_{1,2}(x) (x-1)(x-2) \\
       & + q_{1,1}(x)(x-1) + \phi(1).
\end{align*}
Note that $\phi(x)$ can be re-expressed similarly for any other points $i\in [2,n]$.
Importantly, note that there are only $\Theta(n)$ quotient and accumulator polynomials shared by all $\phi(i)$ expressions, $i\in[n]$.

Our second key idea follows naturally: we commit to all these $\Theta(n)$ quotient polynomials in the multipoint evaluation of $\phi$.
This gives us logarithmic-sized evaluation proofs for any point $i\in[n]$.
We call these proofs \textit{AMT proofs}.
For example, in \cref{f:multipoint-eval}, the AMT proof for $\phi(4)$ would be $\{g^{q_{1,8}(\tau)}, g^{q_{1,4}(\tau)}, g^{q_{3,4}(\tau)}, g^{q_{4,4}(\tau)}\}$, where $\tau$ denotes the trapdoor used in KZG commitments (see \cref{s:prelim:polycommit}).

Recall that a multipoint evaluation of $\phi$ at $n$ points takes $\Theta(n\log^2{n})$ time (see \cref{s:prelim:multipoint-eval}).
This asymptotically dominates the $\Theta(n\log{n})$ time to commit to the quotients.
Thus, \textit{for now}, the time to compute an AMT is $\Theta(n\log^2{n})$.
We explain how to speed this up to $\Theta(\amtDealTime)$ in \cref{s:amt:roots-of-unity} for a carefully selected set of $n$ evaluation points.

\subsection{Verifying our New Evaluation Proofs}
\label{s:amt:verifying-proofs}
The next question is how to verify our new logarithmic-sized AMT proofs.
Recall that, given any point $i$, $\phi(x)$ can be expressed as:
\begin{align}
\label{eq:amt-path}
\phi(x) &= \phi(i) + \sum_{w\in \mathsf{\treepath(i)}} q_w(x) a_w(x)
\end{align}
where $\treepath(i)$ is the set of nodes along the path from the root to $\phi(i)$ and $q_w$ and $a_w$ denote the quotient and accumulator polynomials stored at node $w$ in the multipoint evaluation tree (see \cref{f:multipoint-eval}).
How can we verify a proof $\pi_i = \left(g^{q_w(\tau)}\right)_{w\in\treepath(i)}$ for $\phi(i) = y_i$?
We simply use a bilinear map to check that \cref{eq:amt-path} holds at $x=\tau$:
\begin{align}
\label{eq:amt-verify}
e(g^{\phi(\tau)}, g) &\stackrel{?}{=} e(g^{y_i},g) \prod_{w\in\treepath(i)} e(g^{q_w(\tau)}, g^{a_w(\tau)}) \Leftrightarrow\\ \nonumber
e(g, g)^{\phi(\tau)} &\stackrel{?}{=} e(g,g)^{y_i} \prod_{w\in\treepath(i)} e(g,g)^{q_w(\tau) a_w(\tau)}\Leftrightarrow\\ \nonumber
e(g, g)^{\phi(\tau)} &\stackrel{?}{=} e(g,g)^{{y_i}+\sum_{i\in\treepath(w)} {q_w(\tau) a_w(\tau)}}\Leftrightarrow\\ \nonumber
         \phi(\tau)  &\stackrel{?}{=} {y_i}+\sum_{w\in\treepath(i)} {q_w(\tau) a_w(\tau)}.
\end{align}
This is reminiscent of how KZG proofs are verified by checking that $\phi(x) = q_i(x)(x-i) + \phi(i)$ holds at $x=\tau$ (see \cref{s:prelim:polycommit}).
However, note that \textit{the verifier needs to have the} $g^{a_w(\tau)}$ \textit{accumulator commitments}, which are not part of the AMT proof.
This implies AMT verifiers must have $O(n)$ public parameters, whereas KZG verifiers only need $\{g^\tau\}$ as their public parameters (see \cref{s:prelim:polycommit}).
Fortunately, in \cref{s:amt:public-parameters} we reduce the verifers' public parameters to just $\Theta(\log{t})$.

\subsection{Homomorphic Proofs}
\label{s:amt:homomorphic}
\label{s:scalable-dkg:homomorphic-amt}
%At the end of \ejfdkg's sharing phase, each player must aggregate all his shares, commitments and KZG proofs from the set of qualified players into a final share, commitment and proof (see \cref{alg:dkg}).
%But for this to work in \ourdkg, AMT proofs must be \textit{homomorphic}: $\forall a\in\Fp$, a proof for $f_1(a)$ and a proof for $f_2(a)$ must be \textit{aggregated} into a proof for $(f_1+f_2)(a)$.
AMT proofs are \textit{homomorphic}: a proof for $f_1(j)$ and a proof for $f_2(j)$, can be \textit{aggregated} into a proof for $(f_1+f_2)(j)$.
The intuition for why this holds is that ``adding up'' the multipoint evaluation trees of two polynomials $\phi$ and $\rho$ at the same points (i.e., at $X=\{\omega_N^{j-1}\}_{j\in[n]}$) results in a multipoint evaluation tree of their sum $\phi+\rho$ (also at $X$).
%(The evaluations have to be at the same set of points, say, $\{1,2,\dots,n\}$.)
In more detail, let $q_{w, [\psi]}$ denote the quotient polynomial at node $w$ in $\psi$'s multipoint evaluation tree (at $X$).
Then, one can show that $q_{w,[\phi+\rho]} = q_{w,[\phi]} + q_{w,[\rho]}$ and that $g^{q_{w,[\phi+\rho]}(\tau)} = g^{q_{w,[\phi]}(\tau) + q_{w,[\rho]}(\tau)} = g^{q_{w,[\phi]}(\tau)} g^{q_{w,[\rho]}(\tau)}$.
In other words, given an AMT for $\phi$ and an AMT for $\rho$, we can obtain an AMT for $\phi+\rho$ by multiplying quotient commitments at each node.
It follows that a proof for $f_1(a)$ and one for $f_2(a)$ can be aggregated into a proof for $(f_1+f_2)(a)$ by multiplying commitments at each node.

\subsection{Better AMTs via Roots of Unity}
\label{s:amt:roots-of-unity}
Instead of evaluating $\phi$ at points $\{1,2,3,\dots,n\}$, we assume $n=2^m$ and evaluate $\phi$ at all $n$ $n$th roots of unity in $\Fp$.
Specifically, we compute $\phi(\omega_n^{i-1})$ rather than $\phi(i)$, where $\omega_n$ is a primitive $n$th root of unity.
(We can generalize to any $n$ by using the first $n$ $N$th roots of unity, where $N=2^m$ is the smallest value such that $N\ge n$.)

The main benefit of using roots of unity is they give rise to simpler accumulator polynomials of the form $(x^{2^k} + c)$ in the multipoint evaluation tree (for some $c$).
This speeds up the multipoint evaluation (see \cref{s:amt:proof-time-and-sizes}), since dividing degree-bound $2n$ polynomials by $(x^n + c)$ can be done in $\Theta(n)$ rather than $\Theta(n\log{n})$ time.
In \cref{s:amt:proof-time-and-sizes}, we show this new, optimized AMT proof is $\amtProofSize{t}$ group elements and computing an AMT takes $\Theta(\amtDealTime)$ time.

The $(x^{2^k} + c)$ form of the accumulators is best illustrated with an example.
Let $n=8$ and $\omega_8$ denote a primitive 8th root of unity.
Previously, in \cref{f:multipoint-eval}, the evaluation points $\{1,2,\dots,8\}$ were ordered as $\langle (x-1), (x-2),\dots, (x-8)\rangle$ monomials in the leaves.
Then, the accumulators were computed by multiplying ``up the tree,'' culminating in the root accumulator $\prod_{i\in[8]}(x-i)$.
In our case, the evaluation points are $\{\omega_8^{i-1}\}_{i\in[8]}$ but we reorder them using a bit-reversal permutation~\cite{bitreversal-wiki} as $\langle(x-\omega_8^0), (x-\omega_8^4), (x-\omega_8^2), (x-\omega_8^6), (x-\omega_8^1), (x-\omega_8^5), (x-\omega_8^3), (x-\omega_8^7)\rangle$.
This ordering ensures that, as we multiply ``up the tree,'' all accumulators are of the form $(x^{2^k} + \omega_8^j)$ for some $j$.

Let us see exactly how this happens.
The parent accumulator of the first two leaves $(x-\omega_8^0)$ and $(x-\omega_8^4)$ is their product $(x-\omega_8^0)(x-\omega_8^4) = x^2 - \omega_8^4 x - \omega_8^0 x + \omega_8^0 \omega_8^4$.
Since $\omega_n^i \omega_n^j = \omega_n^{(i+j) \bmod n}$~\cite{CLRS09}, this equals $x^2 - x(\omega_8^4 + \omega_8^0) + \omega_8^4$.
Since $\omega_n^{k+n/2} = -\omega_n^k$~\cite{CLRS09}, this equals $(x^2 + \omega_8^4)$.
The remaining accumulators after $(x^2 + \omega_8^4)$ on this level are $\{(x^2 + \omega_8^0), (x^2 + \omega_8^6), (x^2 + \omega_8^2)\}$.
Recursing on the next level, its accumulators are $\langle(x^4 + \omega_8^4),(x^4 + \omega_8^0)\rangle$.
Finally, the root will be $(x^8-\omega_8^0) = (x^8 - 1) = \prod_{i=0}^{7}(x-\omega_8^i)$.
% bitreverse(000)=000=0
% bitreverse(001)=100=4
% bitreverse(010)=010=2
% bitreverse(011)=110=6
% bitreverse(100)=001=1
% bitreverse(101)=101=5
% bitreverse(110)=011=3
% bitreverse(111)=111=7

% bitreverse(00)=00
% bitreverse(01)=10
% bitreverse(10)=01
% bitreverse(11)=11
%
% (x-w_4^0)(x-w_4^2) = (x^2 - x w_4^0 -x w_4^2 - w_4^2) = (x^2 - x (w_4^0 + w_4^2) - w_4^2).
% recall that \omega_n^{k + n/2} = -\omega_n^k
% since n = 4, we have w_4^{0+2} = -w_4^0
% so we have (x^2 - w_4^2), which equals (x^2 + w_4^0) = (x^2 + 1)
%
% (x-w_4)(x-w_4^3) = (x^2 - x (w_4 + w_4^3) - w_4^4) = (x^2 - x (w_4 + w_4^3) - 1)
% recall that \omega_n^{k + n/2} = -\omega_n^k
% since n = 4, we have w_4^{1 + 2} = - w_4, i.e., w_4^3 = -w_4
% so we have (x^2 - 1)

\subsection{Prover Time and Proof Sizes}
\label{s:amt:proof-time-and-sizes}
We will restrict ourselves to our $n = 2^m$ and $\deg{\phi}=t-1 < n$ setting.
We first show that computing our optimized, roots-of-unity-based AMT takes $O(n\log{t})$ time (see \cref{s:amt:roots-of-unity}).
The key observation is that, when computing the AMT, divisions at higher levels (i.e., closer to the root) in the tree are \textit{trivial} and need not be performed.
Specifically, at sufficiently high levels, the degree of the divisors (i.e., accumulators) are larger than the degrees of the dividends (i.e., remainders), and always give quotients equal to zero.
Since zero quotients can be easily recreated by verifiers, their commitments need not be included in the proof.
We expand on this next.

Let us number levels differently, from $\log{n}$ (the root) to 0 (the leaves), so that level $i$ has $n/2^i$ nodes, each with an accumulator of degree $2^i$.
Now, let $k$ be the smallest value such that $2^k \le \deg{\phi} < 2^{k+1}$.
In other words, $k$ is the level at which accumulator degrees are $\le \deg{\phi}$ and thus divisions are non-trivial.
Put differently, each node on level $k$ will be the root node of an (authenticated) multipoint evaluation (sub)tree.
We argue that the time to compute any one such subtree is $O(2^k \log{2^k})$ and, since there are $n/2^k$ such subtrees, the final AMT takes $O(n\log{2^k}) = O(n\log{t})$ time since $2^k\le t-1=\deg{\phi}$.
We prove this inductively next.

At the root node of a level $k$ subtree, the dividend $d_k = \phi$ has $\deg{d_k} < 2^{k+1}$ (by definition of $k$ above).
The accumulator $a_k$ has $\deg{a_k} = 2^k$.
Thus, the quotient $q_k = d_k/a_k$ will have $\deg{q_k} = \deg{d_k}-\deg{a_k}< 2^{k+1} - 2^k = 2^k$ and the remainder $r_k = d_k \bmod a_k$ will have $\deg{r_k} < \deg{a_k} = 2^k$.
The division at this level will only take $O(\deg{d_k})=O(2^{k+1})$ time, thanks to the $(x^{2^k} + c)$ form of $a_k$.
Committing to the quotient will take $O(2^{k})$ time.
To summarize, at level $k$ we are doing $O(2^{k+1})$ work and $\deg{d_k} < 2^{k+1}, \deg{a_k} = 2^k, \deg{q_k} < 2^k, \deg{r_k} < 2^k$.

Next, we argue that the amount of work \textit{per node} on level $k-1$ is half the work per node at level $k$.
This is because (1) the dividend $d_{k-1}$ is set to the remainder $r_k$ from the parent, so $\deg{d_{k-1}} < 2^k$, (2) $\deg{a_{k-1}}=2^{k-1}$, (3) $\deg{q_{k-1}} = \deg{d_{k-1}}-\deg{a_{k-1}}<2^k-2^{k-1}=2^{k-1}$ and (4) $\deg{r_{k-1}} < \deg{a_{k-1}}=2^{k-1}$.
Thus, at level $k-1$, the division takes $O(2^k)$ time and committing to the quotient takes $O(2^{k-1})$ time.
As a result, the time to compute the subtree can be expressed as $T(2^{k+1}) = 2T(2^{k+1} / 2) + O(2^{k+1}) = O(2^{k}\log{2^{k}})$.

Finally, an AMT proof is $O(\log{t})$-sized.
Recall that quotients in the AMT are non-zero only at levels $k$ and below, where $2^k \le t-1 < 2^{k+1}$.
Thus, an AMT proof will only have non-zero quotients at levels $k, k-1, k-2, \dots, 1, 0$.
Since $k = \floor{\log_2(t-1)}$ the exact proof size is $\floor{\log_2(t-1)}+1$ group elements.


\subsection{Keeping (Almost) the Same Public Parameters}
\label{s:amt:public-parameters}
Do AMTs need extra public parameters, which would impose unwanted overhead in the trusted setup phase (see \cref{s:prelim:trusted-setup})?
Recall that in KZG, given $(t-1)$-SDH public parameters, one can commit to any degree-bound $t$ polynomials and compute \textit{any number} of KZG evaluation proofs.
In contrast, computing an AMT at $n > t-1$ points seems to require committing to degree $n > t-1$ accumulator polynomials (e.g., to the root accumulator $(x^n-1)$).
Yet this is not possible given only $(t-1)$-SDH parameters, as ensured by the $(t-1)$-polyDH assumption (see \cref{def:q-polydh}).

Fortunately, when computing an AMT, divisions by accumulators of degree $> t-1$ always give quotient zero (see \cref{s:amt:proof-time-and-sizes}).
This means that, when pairing such quotients with their accumulators during proof verification, the result will always be $1_{\Group_T}$ (see \cref{eq:amt-verify}).
In other words, such pairings need never be computed and so their corresponding accumulators (of degree $>t-1$) need never be committed to.
Furthermore, quotients are not problematic since they always have degree $< \deg{\phi} = t-1$ (or are equal to zero).

Second, AMT verifiers only need a logarithmic number of $g^{\tau^{2^k}}$ powers to recreate any accumulator commitment $g^{a_w(\tau)}$.
(This is a bit worse than KZG verifiers, who only need $g^\tau$.)
Specifically, given a subset $\{g^{\tau^{2^k}} \mathrel| 0 \le k\le \floor{\log(t-1)}\}$ of the $(t-1)$-SDH parameters, the verifier can commit to any degree-bound $t$ accumulator of the $(x^{2^k}+c)$ form.
Thus, we impose no additional overhead in the trusted setup.
In contrast, if we evaluated $\phi$ at $\{1,2,\dots, n\}$, verifiers would need all $(t-1)$-SDH public parameters to reconstruct the accumulators.