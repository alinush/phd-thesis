\section{How to Share a Public Key}
\label{s:intro:part1}

Almost every cryptographic scheme assumes actors know each other's public keys.
This assumption is often referred to as the \textit{public key infrastructure} (PKI) assumption.
For example, a PKI enables users to safely shop online on Amazon.com, to message each other securely on WhatsApp, and to use social media on LinkedIn.com.
Yet, despite their absolute necessity for security, state of the art PKIs remain inadequate in several ways.

First, PKIs rely on the unreasonable assumption of \textit{trusted parties}.
For example, Facebook, who develops and owns WhatsApp, runs a \textit{public key directory} (PKD) mapping each user's phone number to that user's public key.
This \textit{centralized} control of the PKD by Facebook is quite natural. 
After all, the PKD is a crucial component of WhatsApp's infrastructure and needs incentives to be maintained, incentives which Facebook has.
Unfortunately, this centralized approach assumes we live in a world where Facebook cannot be compromised or coerced.
In reality, mass surveillance revelations tell us the exact opposite story~\cite{EdSnowden,AAB+15}.
Thus, it is not unfathomable for a rogue Facebook to \textit{man-in-the-middle} ``secure'' WhatsApp conversations.
Indeed, this would not be too difficult either.
To snoop on two users, say \textit{\alice} and \textit{\bob}, Facebook need only \textit{equivocate} about their public keys in the PKD.

Second, when trusted parties go rogue, \textit{PKI attacks cannot be easily detected by victims} like Alice and Bob.
We focus on ``detection'' rather than ``prevention,'' since it is impossible to prevent PKI attacks in this strong, pessimistic model where trusted parties can go rogue.
Fortunately, post-attack detection is still possible, although not always easily achieved, as we detail next.

Consider the following (rather contrived) example.
\alice and \bob's WhatsApp conversation is man-in-the-middled by Facebook.
Fortunately, they both ``record'' all of their encrypted WhatsApp traffic.
They then meet in a coffee shop, inspect this traffic and notice that, at some point, their public keys were changed to unrecognized \textit{fake} ones that were subsequently used to encrypt their messages.
% Bob notices in Alice's recorded traffic that his public key was changed.
% Alice notices in Bob's traffic that her public key was changed too.
They then conclude a man-in-the-middle attack occurred.
However, their effort is considerable and does not scale to detecting attacks in multiple conversations.
Thus, more efficient detection methods are needed.

Third, (some) PKIs suffer from a \textit{weakest link in the chain} problem.
In other words, while one trusted party is an unreasonable assumption, $n>1$ trusted parties is much worse, since it increases the attack surface.
Sadly, the PKI used to secure HTTPS connections is the best example of this.
The \textit{Web PKI} is comprised of $1400$+ \textit{Certificate Authorities} (CAs)~\cite{ssliverse-talk} which digitally sign \textit{certificates} that map a website to its public key.
Quite terrifyingly, \textit{any} CA, in \textit{any} country, can sign a certificate for \textit{any} website.
Not surprisingly, this has led to many websites being surreptitiously \textit{impersonated}~\cite{cahacks,cahacksurvey}.
Even worse, it is conceivable that some impersonation attacks went undetected.
Fortunately, in 2013, Google took a key step towards \textit{detecting} and thus \textit{deterring} such attacks~\cite{ct}, which we discuss next.

\paragraph{Maintaining Trust in PKIs with Transparency Logs.}
In 2011, Google was the victim of two impersonation attacks~\cite{mitmgoogle,mitmgoogle2}.
As a result, in March 2013, Google deployed \textit{Certificate Transparency} (CT)~\cite{ct} as a way of detecting impersonation attacks and, hopefully, deterring them by publicly (though not provably) revealing their traces.
The key idea behind CT is to \textit{require} that all certificates for websites be published in a \textit{transparency log}.
In other words, unless a certificate comes with an \textit{inclusion proof} in the transparency log, a browser such as Firefox will never accept that certificate.
This way, if a rogue CA issues a fake certificate for a victim website, then the victim can detect this attack by searching or \textit{monitoring} in the transparency log.
This immediately raises the question of who runs this log and what if, just like a CA, it misbehaves?
The answer is an untrusted \textit{log server} runs the log but, if it misbehaves, it can be caught either (1) by \textit{domain owners}  such as \texttt{google.com},  \texttt{yahoo.com} or \texttt{your-website.com}, or (2) by users like \alice and \bob, or (3) by any other third-party \textit{verifier}.
Importantly, this misbehavior can be \textit{cryptographically proven}.

It is important to understand that CT logs cannot \textit{prevent} impersonation attacks, since a rogue CA can always (``kamikaze-style'') publish a fake certificate for, say, \texttt{google.com} in the log which will be used by (unsuspecting) users, until the fake certificate is detected and revoked by Google.
Instead, CT logs make such attacks quickly \textit{detectable} by the impersonated victim.
Specifically, the victim can now search the log for its certificates (i.e., monitor) and discover the fake certificate.
Unfortunately, this does not prove the certificate is ``fake.''
After all, the CA can falsely claim the victim was the one who asked for this certificate to be issued.
Since there is no cryptographic way of figuring out the truth, this must be resolved in the real world via the judicial system.
This way, incompetent or malicious CAs can be eliminated from the Web PKI, leading to a more trustworthy ecosystem.
Furthermore, the threat of legal action should deter CAs from misbehaving.
Yet this is only possible if logs are honest, which brings us to the focus of \cref{s:aads} of this thesis: \textit{cryptography for transparency logs.}

\paragraph{Transparency Logs with Succinct Proofs.}
To operate correctly, transparency logs must have three key properties.
First, when a browser connects to a website and receives a certificate, the log server should prove to the browser that the certificate is included in the log via an \textit{inclusion proof}.
Otherwise, instead of being forced to publish fake certificates in the log and risk detection, rogue CAs can directly (and thus undetectably) present fake certificates to victim browsers.

Second, when a domain owner \textit{looks up} their own certificates in the log, the server should prove that the domain's complete set of certificates has been returned via a \textit{lookup proof}.
Otherwise, if a fake certificate can be left out of the answer, then it would never be discovered by victims monitoring the log, leading to undetected impersonation attacks.

Third, logs should prove to everybody that they remain append-only via \textit{append-only proofs}.
Otherwise, a fake certificate for a victim website can be inserted in the log, shown to a victim browser and removed quickly, before the victim website has a chance to monitor its certificates.

Unfortunately, although all transparency log designs support these proofs, the proof sizes are too large ~\cite{ct,coniks,Ryan2014}: either the lookup proof or the append-only proof is linear in the size of the log.
As a workaround to obtain small proof sizes, all transparency logs introduce additional trust assumptions or additional actors.
Specifically, they either introduce additional \textit{verifiers} who check the log for misbehavior on behalf of users~\cite{ct,Ryan2014}, or rely on non-collusion assumptions~\cite{aki,arpki,policert}, or require users to \textit{frequently check the log}~\cite{Ryan2014,dtki,coniks}.
As a result, such constructions are either difficult to deploy~\cite{aki,arpki,policert}, or expensive in terms of communication overhead~\cite{ct,coniks}, or slow to detect attacks~\cite{Ryan2014,dtki}.

In contrast, logs with succinct proofs do not suffer from these problems.
First, they require no additional verifiers, making them easier to deploy.
Second, by definition, they have low communication overhead.
Third, since proof sizes are small, they can be frequently queried to detect attacks fast.
Thus, the first question this thesis asks is:

\newcommand{\qoneprefix}{\textbf{\underline{Question 1}:}\xspace}
\newcommand{\qonesuffix}{Can we build transparency logs with succinct lookup proofs\xspace}
\newlength{\qonelength}
\settowidth{\qonelength}{\qoneprefix\qonesuffix}
\begin{center}
\parbox{\qonelength}{%
    \centering
    \qoneprefix \textit{\qonesuffix and succinct append-only proofs in the single, untrusted sever model?}
}
\end{center}

We answer this question positively in \cref{s:aads}.
We give constructions for logs that map application-specific \textit{keys} to one or more application-specific \textit{values}, in an append-only fashion.
Such logs are ideal for building public-key directories (i.e., a ``key'' corresponds to a domain and the ``values'' correspond to the certificates of that domain).
We also give constructions for logs that maintain an append-only set of application-specific \textit{elements}.
Such logs are ideal for keeping track of revoked certificates~\cite{Laurie15}.
Our transparency logs are useful not just for securing PKIs, but also for securing software distribution~\cite{at,contour,chainiac,software-dist-transparency,TD17,STV+16}.
Although our constructions offer small proof sizes, they are not yet fast enough to append to, at least not at the scale of a realistic PKI.
We believe this can be addressed by future work and we discuss possible directions in \cref{s:future-work:aads}.

\section{How to Share a Secret (With Two Million People)}
Many cryptographic schemes require that some information be kept secret from the adversary.
For example, in public-key encryption schemes~\cite{rsa,DH76}, \alice and \bob have to keep their \textit{private keys} secret, or else the adversary can snoop on their conversation.
Similarly, in cryptocurrencies like Bitcoin~\cite{bitcoin}, users have to keep their private keys secret, or else the adversary can steal their coins.
Yet this requirement of keeping a private key secret from the adversary is in constant tension with the need to use it (and thus temporarily expose it) for achieving cryptographic goals: e.g., sending an encrypted message, or transferring coins.
Fortunately, one way to address this tension is via \textit{threshold cryptosystems}.

In a threshold cryptosystem, the private key is \textit{split up} amongst a set of $n$ players, such that \textit{only} subsets of size $\ge t$ players can collaborate to use the key.
Threshold cryptosystems better hide the key from the adversary, who now has to compromise at least $t$ players.
At the same time, they preserve functionality by allowing the players to collaborate to use the key.
Importantly, players can use the key \textit{without ever exposing it}~\cite{Desmedt87,DF90,DesmedtFrankel1992SharedGeneration}.

\paragraph{Threshold Signatures.}
For example, a digital signature scheme such as RSA~\cite{rsa} or BLS~\cite{BLS04} can be transformed into a \textit{threshold signature scheme} (TSS) by splitting up its private key amongst $n$ \textit{signers}.
This way, each signer has a \textit{share} of the private key and only subsets of $\ge t$ signers can collaborate to produce a signature.
Importantly, the signers can do this \textit{without} reconstructing the private key ``in plain sight'' and thus exposing it.

In a TSS, signing a message $m$ is an interactive protocol.
Each signer is given $m$ and computes a \textit{signature share}, which is just a signature on $m$ but under that signer's share of the private key.
Then, any $\ge t$ signers get together and ``aggregate'' their signature shares into the \textit{final signature} on $m$.
As a result, the private key is never exposed but can still be used to sign!

TSS schemes have many important applications.
First, they can be used to decentralize Certificate Authorities (CAs) in PKIs (see \cref{s:intro:part1}) by distributing their extremely-sensitive private key amongst many sub-CAs~\cite{STV+16}.
Second, they can be used to create very simple but resilient \textit{randomness beacons}~\cite{randherd,CD17a,ouroboros}, which have many applications in cryptography~\cite{zcash-mpc2}.
Third, they can be used to create simple voting protocols and thus to scale consensus protocols~\cite{GAG+19}.

While applications such as decentralizing Certificate Authorities (CA) do not require a high number of signers, other applications such as voting could require millions of signers.
Unfortunately, naively-implemented TSS schemes do not scale to such a large number of signers.
This is because all current TSS scheme implementations use a naive polynomial interpolation algorithm~\cite{BT04}, which runs in time quadratic in the number of signers $n$.
Thus, the second question this thesis asks is:

\newcommand{\qtwoprefix}{\textbf{\underline{Question 2}:}\xspace}
\newcommand{\qtwosuffix}{Can we scale threshold signature schemes to millions of signers?}
\newlength{\qtwolength}
\settowidth{\qtwolength}{\qtwoprefix\qtwosuffix}
\begin{center}
\parbox{\qtwolength}{%
    \centering
    \qtwoprefix \textit{\qtwosuffix}
}
\end{center}

We answer this question positively in \cref{s:threshsig:fast-lagr}.
We adapt existing, quasilinear-time interpolation algorithms~\cite{vG13ModernCh10} to work with discrete log-based signature schemes such as BLS~\cite{BLS04}.
This gives us a dramatic speed-up over naive, quadratic-time interpolation algorithms.
Specifically, our threshold BLS signature~\cite{Boldyreva03} can aggregate a signature from one million signers in \blsEffTimeFriendly{2097151}, a drastic improvement over \blsNaiveTimeFriendly{2097151}, if done using a naive, quadratic-time interpolation algorithm.

\paragraph{Bootstrapping Threshold Signatures.}
Yet a key challenge still remains: even if TSS schemes scaled to millions of signers, how can such a large-scale TSS be securely bootstrapped?
In other words, how will the private key be split up amongst the $n$ signers so that nobody learns it but everybody has a share of it?
If a single \textit{trusted dealer} were to generate the private key and split it amongst the signers using \textit{secret sharing}~\cite{Shamir79,CGMA85}, then he would be trusted to forget the private key.
This dealer would thus become a single point of failure, which defeats the point of a threshold cryptosystem.

Fortunately, \textit{distributed key generation} (DKG) protocols~\cite{Ingemarsson1991,Pedersen1991AThreshold,GJKR07,Kate2010} can remove this single point of failure when bootstrapping a TSS.
In a DKG, the $n$ signers jointly and randomly pick the private key and split it amongst each other, without any signers learning the final private key.
In other words, the $n$ signers can be used to ``implement'' an always-honest dealer.
Unfortunately, DKG protocols can be computationally-expensive, requiring computation quadratic in the number of signers $n$~\cite{Kate2010}.
Thus, the second question this thesis asks is:

\newcommand{\qthreeprefix}{\textbf{\underline{Question 3}:}\xspace}
\newcommand{\qthreesuffix}{Can we scale distributed key generation (DKG) protocols\xspace}
\newlength{\qthreelength}
\settowidth{\qthreelength}{\qthreeprefix\qthreesuffix}
\begin{center}
\parbox{\qthreelength}{%
    \centering
    \qthreeprefix \textit{\qthreesuffix for discrete log-based cryptosystems to millions of players?}
}
\end{center}

We answer this question positively in \cref{s:threshcrypto}.
First, we present a novel DKG protocol, which can generate keys for TSS schemes with millions of players in hours, rather than a year (see \cref{s:eval:dkg}).
In this process, we also introduce a scalable Verifiable Secret Sharing (VSS) protocol~\cite{Shamir79,CGMA85,Feldman87} and a novel Vector Commitment (VC) scheme (see \cref{s:vcs:from-amt}).
While VSS is a key component of DKG protocols, VCs are of independent interest and have applications to stateless cryptocurrencies~\cite{CPZ18,BBF19}.
Second, we present a threshold-variant of BLS signatures~\cite{BLS04,Boldyreva03} that can aggregate a signature from one million signers in \blsEffTime{2097151}, a drastic improvement over \blsNaiveTime{2097151}, if done naively.

\section{Organization}
\label{s:intro:related-work}
This thesis is organized into two parts.
The necessary, common background is in \cref{s:prelim}.

\paragraph{\cref{s:aads}.}
This part covers our append-only authenticated data structures for transparency logs~\cite{ct,Ryan2014} with application to public-key distribution.
In \cref{s:aads:intro}, we introduce transparency logs, motivate the need to build communication-efficient logs without trusted third parties and discuss related work.
In \cref{s:prelim:aads}, we give additional background on \textit{cryptographic accumulators}~\cite{acc-rsa,Nguyen05}, which our data structures are built upon.

In \cref{s:aas}, we introduce \textit{append-only authenticated sets (AAS)}, which can be used to build \textit{revocation transparency (RT)} logs~\cite{Laurie15}.
First, in \cref{s:aas:defs} we formalize AAS correctness and security definitions.
Second, in \cref{s:aas:from-bilinear-acc}, we construct an AAS from \textit{bilinear accumulators}~\cite{Nguyen05}.
Third,  in \cref{s:aas:from-rsa-acc} we construct an AAS from \textit{RSA accumulators}~\cite{acc-rsa}.
Finally, we prove these two constructions are secure in \cref{s:aas:from-bilinear-acc:proofs} and \cref{s:aas:from-rsa-acc:proofs}, respectively.

In \cref{s:aad}, we introduce \textit{append-only authenticated dictionaries (AADs)}, which can be used to build general-purpose transparency logs~\cite{ELC16,trillian}.
First, in \cref{s:aad:defs}, we formalize AAD correctness and security definitions.
Second, in \cref{s:aad:from-acc}, we show how to build an AAD from any cryptographic accumulator by slightly modifying our AAS constructions from \cref{s:aas}.
Third, in \cref{s:aad:from-bilinear-acc:eval}, we implement and evaluate our bilinear-based AAD construction.

\paragraph{\cref{s:threshcrypto}.}
This part covers our scalable threshold cryptosystems: threshold signature schemes (TSS), verifiable secret sharing (VSS) protocols and distributed key generation (DKG) protocols.
In \cref{s:threshcrypto:intro}, we introduce threshold cryptosystems, motivate their need to scale and discuss related work.
In \cref{s:prelim:threshcrypto}, we give additional background on TSS, VSS and DKG protocols.

In \cref{s:amt}, we introduce \textit{authenticated multipoint evaluation trees (AMTs)}, a new way of precomputing evaluation proofs in polynomial commitments~\cite{KZG10a}.
AMTs will be the key ingredient for scaling VSS and DKG protocols later on.
In \cref{s:vcs:from-amt}, we give a new \textit{vector commitment (VC)} scheme based on AMTs.

In \cref{s:threshcrypto:scalable-systems}, we give scalable constructions for TSS, VSS and DKG protocols.
First, in \cref{s:threshsig}, we show how a faster algorithm for polynomial interpolation can be used to scale threshold signatures to millions of signers.
%(and any threshold cryptosystems based on Shamir secret sharing~\cite{Shamir79}).
Second, in \cref{s:scalable-vss}, we use AMTs to scale VSS protocols to millions of players.
Third, in \cref{s:scalable-dkg}, we use our scalable VSS to scale DKG protocols.
Finally, in \cref{s:threshcrypto:eval}, we implement and evaluate our scalable TSS, VSS and DKG protocols.

\paragraph{Future Work and Conclusion.}
In \cref{s:future-work:aads}, we give directions for future work on AADs, discussing both how to improve our current constructions and how to develop new ones.
In \cref{s:future-work:threshcrypto}, we give directions for future work on threshold cryptosystems, with an emphasis on reducing communication in VSS and DKG protocols.
We conclude in \cref{s:conclusion}.

\section{Publications}
The results of this thesis are also published in two conference papers~\cite{TBP+19,TCZ+20}:

\begin{center}
\textit{``Transparency Logs via Append-only Authenticated Dictionaries''},
Alin Tomescu, Vivek Bhupatiraju, Dimitrios Papadopoulos, Charalampos Papamanthou, Nikos Triandopoulos and Srinivas Devadas, in ACM CCS 2019
\end{center}

\begin{center}
\textit{``Towards Scalable Threshold Cryptosystems''}, Alin Tomescu, Robert Chen, Yiming Zheng, Ittai Abraham, Benny Pinkas, Guy Golan Gueta and Srinivas Devadas, in IEEE S\&P 2020
\end{center}