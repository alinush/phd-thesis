The key difference between AADs and previous work~\cite{BuldasLaudLipmaa2000,ct,Ryan2014,aki,arpki,policert,coniks,dtki} is that we offer succinct proofs for everything while only relying on a single, untrusted log server.
In contrast, previous work either has large proofs~\cite{ct,coniks}, requires users to ``collectively'' verify the log~\cite{Ryan2014,dtki} (which assumes enough honest users and can make detection slow), or makes some kind of trust assumption about one or more actors~\cite{ct,aki,arpki,policert}.
On the other hand, previous work only relies on collision-resistant hash functions (CRHFs), which makes it computationally much cheaper.
However, since communication is more expensive than computation, this is not always the right trade-off.

In contrast, our bilinear constructions require trusted setup, large public parameters, and non-standard assumptions.
Our RSA-based constructions eliminate some of these drawbacks, requiring no trusted setup and having constant-sized public parameters.
Unlike previous work, our constructions are not yet practical due to their high append times and memory usage (see \cref{s:eval:append-time,s:eval:memory}).
Finally, previous work~\cite{aki,arpki,dtki,policert} addresses in more depth the subtleties of log-based PKIs, while our work is focused on improving the transparency log primitive itself by providing succinct proofs with no trust assumptions.

\paragraph{(E)CT.}
Early work proposes the use of Merkle trees for public-key distribution but does not tackle the append-only problem, only offering succinct lookup proofs~\cite{crt,certificate-rev-upd,BuldasLaudLipmaa2000}.
Accumulators are dismissed in~\cite{BuldasLaudLipmaa2000} due to trusted setup requirements.
Certificate Transparency (CT)~\cite{ct} provides succinct append-only proofs via \textit{history trees} (HTs).
Unfortunately, CT does not offer succinct lookup proofs, relying on users to download each update to the log to discover fake PKs, which can be bandwidth-intensive (see \cref{s:eval:worth-it}).
Alternatively, users can look up their PKs via one or more CT \textit{monitors}, who download and index the entire log.
But this introduces a trust assumption that a user can reach at least one honest CT monitor.
Enhanced Certificate Transparency (ECT) addresses CT's shortcomings by combining a lexicographic tree with a chronologic tree, with collective verification by users.
As discussed above, collective verification assumes enough honest users exist and can make detection of append-only breaks slow.
Alternatively, ECT can also rely on one or more ``public auditors'' to verify correspondence of the two trees, but this introduces a trust assumption.

\paragraph{A(RP)KI and PoliCert.}
Accountable Key Infrastructure (AKI)~\cite{aki} introduces a checks-and-balances approach where log servers manage a lexicographic tree of certificates and so-called ``validators'' ensure log servers update their trees in an append-only fashion.
Unfortunately, AKI must \textit{``assume a set of entities that do not collude: CAs, public log servers, and validators''}~\cite{aki}. 
At the same time, an advantage of AKI is that validators serve as nodes in a gossip protocol, which helps detect forks.
% PoliCert paper: "the achieved property is that with successfully registered SCP, an adversary even with n - 1 parties compromised, cannot launch impersonation attack undetectably, as n parties are actively involved in asserting correctness of SCPs and MSCs."
ARPKI~\cite{arpki} and PoliCert~\cite{policert} extend AKI by providing security against attackers controlling $n-1$ out of $n$ actors.
Unfortunately, this means ARPKI and PoliCert rely on an anytrust assumption to keep their logs append-only.
On the other hand, AKI, ARPKI and PoliCert carefully consider many of the intricacies of PKIs in their design (e.g., certificate policies, browser policies, deployment incentives, interoperability).
In addition, ARPKI formally verifies their design.

\paragraph{CONIKS and DTKI.}
CONIKS~\cite{coniks} periodically appends a digest of a prefix tree to a hash chain.
However, users must collectively verify the tree remains append-only.
Specifically, \textit{in every published digest}, each user checks that their own public key has not been removed or maliciously changed.
Unfortunately, this process can be bandwidth-intensive when digests are published frequently (see \cref{s:eval:worth-it}).
DTKI~\cite{dtki} observes that relying on a multiplicity of logs (as in CT) creates overhead for domain owners who must check for impersonation in every log.
DTKI introduces a \textit{mapping log} that associates sets of domains to their own exclusive transparency log.
Unfortunately, like ECT, DTKI also relies on users to collectively verify its many logs.
To summarize, while previous work~\cite{aki,arpki,policert,dtki} addresses many facets of the transparent PKI problem, it does not address the problem of building a transparency log with succinct proofs without trust assumptions and without collective verification.

\paragraph{Byzantine Fault Tolerance (BFT).}
If one is willing to move away from the single untrusted server model, then a transparency log could be implemented using BFT protocols~\cite{Lamport1982TheByzantine,pbft,bitcoin}.
In fact, BFT can trivially keep logs append-only and provide lookup proofs via sorted Merkle trees.
With permissioned BFT~\cite{pbft}, one must trust that two thirds of BFT servers are honest.
While we are not aware of permissioned implementations, they are worth exploring.
For example, in the key transparency setting, it is conceivable that CAs might act as BFT servers.
With permissionless BFT~\cite{bitcoin,ethereum}, one needs a cryptocurrency secured by proof-of-work or proof-of-stake.
Examples of this are Namecoin~\cite{namecoin}, Blockstack~\cite{blockstack} and EthIKS~\cite{ethiks}.

\paragraph{Formalizations.}
Previous work formalizes Certificate Transparency (CT)~\cite{transparency-overlays,secure-logging-schemes-and-ct} and general transparency logs~\cite{transparency-overlays}.
In contrast, our work formalizes append-only authenticated dictionaries (AAD) and sets (AAS), which can be used as transparency logs.
Our AAD abstraction is more expressive than the \textit{dynamic list commitment (DLC)} abstraction introduced in~\cite{transparency-overlays}.
Specifically, DLCs are append-only lists with non-membership by insertion time, while AADs are append-only dictionaries with non-membership by arbitrary keys.
Nonetheless, AADs can easily associate each key-value pair to its insertion time securely via Merkle aggregation~\cite{ht}.
This way, AADs can easily support non-membership by insertion time as well.
Finally, previous work carefully formalizes proofs of misbehavior for transparency logs~\cite{transparency-overlays,secure-logging-schemes-and-ct}.
Although misbehavior in AADs is provable too, we do not formalize this in the paper.

Neither our work nor previous work adequately models the network connectivity assumptions needed to detect forks in a gossip protocol.
Lastly, previous work improves or extends transparency logging in various ways but does not tackle the append-only problem~\cite{ct-with-privacy,insynd,lwm}.
