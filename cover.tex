% -*-latex-*-
% 
% For questions, comments, concerns or complaints:
% thesis@mit.edu
% 
%
% $Log: cover.tex,v $
% Revision 1.8.1 2015/08/14 8:58:00  alinush
% Citations in .tex files now point to dummy entries in main.bib
% Font smoothness issues solved. Created an easy to use Makefile.
%
% Revision 1.8  2008/05/13 15:02:15  jdreed
% Degree month is June, not May.  Added note about prevdegrees.
% Arthur Smith's title updated
%
% Revision 1.7  2001/02/08 18:53:16  boojum
% changed some \newpages to \cleardoublepages
%
% Revision 1.6  1999/10/21 14:49:31  boojum
% changed comment referring to documentstyle
%
% Revision 1.5  1999/10/21 14:39:04  boojum
% *** empty log message ***
%
% Revision 1.4  1997/04/18  17:54:10  othomas
% added page numbers on abstract and cover, and made 1 abstract
% page the default rather than 2.  (anne hunter tells me this
% is the new institute standard.)
%
% Revision 1.4  1997/04/18  17:54:10  othomas
% added page numbers on abstract and cover, and made 1 abstract
% page the default rather than 2.  (anne hunter tells me this
% is the new institute standard.)
%
% Revision 1.3  93/05/17  17:06:29  starflt
% Added acknowledgements section (suggested by tompalka)
% 
% Revision 1.2  92/04/22  13:13:13  epeisach
% Fixes for 1991 course 6 requirements
% Phrase "and to grant others the right to do so" has been added to 
% permission clause
% Second copy of abstract is not counted as separate pages so numbering works
% out
% 
% Revision 1.1  92/04/22  13:08:20  epeisach

% NOTE:
% These templates make an effort to conform to the MIT Thesis specifications,
% however the specifications can change.  We recommend that you verify the
% layout of your title page with your thesis advisor and/or the MIT 
% Libraries before printing your final copy.
\title{\mytitle}

\author{Ioan Alin Tomescu Nicolescu}
% If you wish to list your previous degrees on the cover page, use the 
% previous degrees command:
       \prevdegrees{
        B.S., Stony Brook University (2012)\\
        S.M., Massachusetts Institute of Technology (2015)}
\department{Department of Electrical Engineering and Computer Science}

% If the thesis is for two degrees simultaneously, list them both
% separated by \and like this:
\degree{Doctor of Philosophy in Computer Science}% \and Master of Science}
% If the thesis is for a Masters degree:
% \degree{Master of Science in Electrical Engineering and Computer Science}
% If this is a Bachelor's degree thesis:
% \degree{Bachelor of Science in Computer Science and Engineering}
% \degree{Master of Science in Electrical Engineering and Computer Science}

% As of the 2007-08 academic year, valid degree months are September, 
% February, or June.  The default is June.
\degreemonth{February}
\degreeyear{2020}
\thesisdate{January 28, 2020}

%% By default, the thesis will be copyrighted to MIT.  If you need to copyright
%% the thesis to yourself, just specify the `vi' documentclass option.  If for
%% some reason you want to exactly specify the copyright notice text, you can
%% use the \copyrightnoticetext command.  
%\copyrightnoticetext{\copyright IBM, 1990.  Do not open till Xmas.}

% If there is more than one supervisor, use the \supervisor command
% once for each.
\supervisor{Srinivas Devadas}{Edwin Sibley Webster Professor of Electrical Engineering\\and Computer Science}

% This is the department committee chairman, not the thesis committee
% chairman.  You should replace this with your Department's Committee
% Chairman.
\chairman{Leslie A. Kolodziejski}{Professor of Electrical Engineering and Computer Science\\Chair, Department Committee on Graduate Students}

% Make the titlepage based on the above information.  If you need
% something special and can't use the standard form, you can specify
% the exact text of the titlepage yourself.  Put it in a titlepage
% environment and leave blank lines where you want vertical space.
% The spaces will be adjusted to fill the entire page.  The dotted
% lines for the signatures are made with the \signature command.
\maketitle

% The abstractpage environment sets up everything on the page except
% the text itself.  The title and other header material are put at the
% top of the page, and the supervisors are listed at the bottom.  A
% new page is begun both before and after.  Of course, an abstract may
% be more than one page itself.  If you need more control over the
% format of the page, you can use the abstract environment, which puts
% the word "Abstract" at the beginning and single spaces its text.

%% You can either \input (*not* \include) your abstract file, or you can put
%% the text of the abstract directly between the \begin{abstractpage} and
%% \end{abstractpage} commands.

% First copy: start a new page, and save the page number.
\cleardoublepage
% Uncomment the next line if you do NOT want a page number on your
% abstract and acknowledgments pages.
% \pagestyle{empty}
\setcounter{savepage}{\thepage}

\begin{abstractpage}
% $Log: abstract.tex,v $
% Revision 1.1  93/05/14  14:56:25  starflt
% Initial revision
% 
% Revision 1.1  90/05/04  10:41:01  lwvanels
% Initial revision
% 
%
%% The text of your abstract and nothing else (other than comments) goes here.
%% It will be single-spaced and the rest of the text that is supposed to go on
%% the abstract page will be generated by the abstractpage environment.  This
%% file should be \input (not \include 'd) from cover.tex

Despite 40+ years of amazing progress, cryptography is constantly plagued by two simple problems: keeping secret keys \textit{secret} and making public keys \textit{public}.
For example, public-key encryption is secure only if each user (1) keeps his secret key out of the hands of the adversary and (2) correctly distributes his public key to all other users.
This thesis seeks to address these two fundamental problems.

First, we introduce communication-efficient, fully-untrusted append-only logs, which can be used to correctly distribute public keys. 
Our constructions have logarithmic-sized proofs for the two key operations in append-only logs: looking up public keys and verifying the log remained append-only. 
In contrast, previous logs either have linear-sized proofs or need extra trust assumptions.
%Although our constructions offer small proof sizes, they are not as fast to append to as Merkle-based logs.
Our logs can also be used to secure software distribution and, we hope, to increase transparency in any institution that wants to do so.

Second, we speed up threshold cryptosystems, which protect secret keys by splitting them up across many users.
We introduce threshold signatures, verifiable secret sharing and distributed key generation protocols that can scale to millions of users. 
Our protocols drastically reduce execution time, anywhere from 2\texttimes\xspace to 4500\texttimes, depending on the scale. 
For example, at large scales, we reduce time from tens of hours to tens of seconds.

At the core of most of our contributions lie new techniques for computing evaluation proofs in constant-sized polynomial commitments.
Specifically, we show how to decrease the time to compute $n$ proofs for a degree-bound $n$ polynomial from $O(n^2)$ to $O(n\log{n})$, at the cost of increasing proof size from $O(1)$ to $O(\log{n})$. 
Our techniques could be of independent interest, as they give rise to other cryptographic schemes, such as Vector Commitments (VCs).


\end{abstractpage}

% Additional copy: start a new page, and reset the page number.  This way,
% the second copy of the abstract is not counted as separate pages.
% Uncomment the next 6 lines if you need two copies of the abstract
% page.
% \setcounter{page}{\thesavepage}
% \begin{abstractpage}
% % $Log: abstract.tex,v $
% Revision 1.1  93/05/14  14:56:25  starflt
% Initial revision
% 
% Revision 1.1  90/05/04  10:41:01  lwvanels
% Initial revision
% 
%
%% The text of your abstract and nothing else (other than comments) goes here.
%% It will be single-spaced and the rest of the text that is supposed to go on
%% the abstract page will be generated by the abstractpage environment.  This
%% file should be \input (not \include 'd) from cover.tex

Despite 40+ years of amazing progress, cryptography is constantly plagued by two simple problems: keeping secret keys \textit{secret} and making public keys \textit{public}.
For example, public-key encryption is secure only if each user (1) keeps his secret key out of the hands of the adversary and (2) correctly distributes his public key to all other users.
This thesis seeks to address these two fundamental problems.

First, we introduce communication-efficient, fully-untrusted append-only logs, which can be used to correctly distribute public keys. 
Our constructions have logarithmic-sized proofs for the two key operations in append-only logs: looking up public keys and verifying the log remained append-only. 
In contrast, previous logs either have linear-sized proofs or need extra trust assumptions.
%Although our constructions offer small proof sizes, they are not as fast to append to as Merkle-based logs.
Our logs can also be used to secure software distribution and, we hope, to increase transparency in any institution that wants to do so.

Second, we speed up threshold cryptosystems, which protect secret keys by splitting them up across many users.
We introduce threshold signatures, verifiable secret sharing and distributed key generation protocols that can scale to millions of users. 
Our protocols drastically reduce execution time, anywhere from 2\texttimes\xspace to 4500\texttimes, depending on the scale. 
For example, at large scales, we reduce time from tens of hours to tens of seconds.

At the core of most of our contributions lie new techniques for computing evaluation proofs in constant-sized polynomial commitments.
Specifically, we show how to decrease the time to compute $n$ proofs for a degree-bound $n$ polynomial from $O(n^2)$ to $O(n\log{n})$, at the cost of increasing proof size from $O(1)$ to $O(\log{n})$. 
Our techniques could be of independent interest, as they give rise to other cryptographic schemes, such as Vector Commitments (VCs).


% \end{abstractpage}

\cleardoublepage

\ifHideAcks

\else
    \clearpage

\newlength\longest

\thispagestyle{empty}
\null\vfill

\settowidth\longest{\large\itshape If you can dream... and not make dreams your master,}
% If you can think... and not make thoughts your aim,
% If you can meet with \textit{Triumph} and \textit{Disaster},
% And treat those two imposters just the same.

\begin{center}
\parbox{\longest}{%
  \raggedright{\large\itshape%
  \par\bigskip
      ``Kindly let me help you or you'll drown,\\
      said the monkey, putting the fish safely up a tree.''\par\bigskip
  }
  \raggedleft\large\MakeUppercase{Alan Watts}\par%
}
\end{center}

\vfill\vfill

\cleardoublepage

    \section*{Acknowledgments}

There are so many people I want to thank, it would be understandable if you get tired of reading this...\\

My advisor, \textbf{Srini Devadas}, for supporting and guiding me during these years.
For the freedom to explore any topic I was interested in.
For his guidance when I did not know what to do with that freedom.
For trusting me to mentor four high school students.
For all the pool parties.
For all the ice cream trips to Ghirardelli's!

My thesis committee, \textbf{Ron Rivest} and \textbf{Vinod Vaikuntanathan}, for their invaluable feedback.

My undergraduate advisor, \textbf{Radu Sion}, for the time and energy he put into mentoring me.
For his constant encouragement.
For his occasionally harsh reality checks.

My research group at MIT.
For inspiring and encouraging me along the way.
For being the most unlikely combination of characters.
\textbf{Ling Ren}, for all the research discussions in our office.
\textbf{Victor Costan}, for reminding me not to stress out and have some fun. (And, for sometimes forcing me to.)
\textbf{Christopher Fletcher}, for once mentioning in passing how awesome polynomials are. (Hence, this thesis.)
\textbf{Quan Nguyen}, for showing us how to build CPUs in Minecraft.
\textbf{Sabrina Neuman}, for teaching us about robots.
\textbf{Xiangyao Yu}, for giving the lab's newest grad student a computer to work on.
\textbf{Hsin-Jung Yang}, for all the FPGA talks.
\textbf{Hanshen Xiao}, for being the math wizard one can always rely on.
\textbf{Charles Herder}, for being the wizard of all wizards.
\textbf{Albert Kwon}, for inspiring me to do work as awesome as his.
\textbf{Jun Wan}, for all the breakfasts at Darwin's and for always checking my math.
\textbf{Kyle Hogan}, for teaching me how to ``coffee'' right.
\textbf{Jules Drean}, for his awesome energy.
\textbf{Sacha Servan-Schreiber}, for walking the Appalachian trail and telling us about it.
\textbf{Alex (Yu) Xia}, for all the times he popped up into our office to check how we're doing.
\textbf{Zack Newman}, for all the cool crypto talks and for his crazy-good life habits.
\textbf{Ilia Lebedev}, for being a generally questionable individual.
For all those times he filled in as a TA for me in Spring 2014.
For all the cocktails he made at Srini's pool parties.
For being crazy enough to buy his own boat. 
Twice.
For all the sailing trips.
(Ilusha, I'm sorry I never got to go on the new boat.)
For all the crazy parties together.
For all of his ``keynote talks'' about Sable Island.
For the 6 years of sharing an office that was never boring, sometimes to the detriment of our research goals.
(Really, Ilusha deserves his own section.)

\textbf{Sally O. Lee}, for all the jokes and laughter.
For having the most hilarious office wall.
For lending me her mother's lovely, Christmas-themed coffee mug, which made all of my mornings in the Stata Center feel so cozy.

My four high school students from the MIT PRIMES program, \textbf{Vivek Bhupatiraju}, \textbf{Robert Chen}, \textbf{John Kuszmaul} and \textbf{Yiming Zheng}, for the joy of creating and solving problems together.
This thesis is based on joint work with them.

My other collaborators, for making all of this work possible.
\textbf{Peter Williams}, for showing me what an ``Aha!'' moment looks like while he worked out the file system design for PrivateFS.
For trusting me to implement this design.
\textbf{Mashael AlSabah}, for the fun discussions and collaboration around email security.
\textbf{Nikos Triandopoulos}, for opening up his door to me when I was a lost graduate student.
For the wonderful collaboration that ensued.
\textbf{Dimitris Papadopoulos}, for all the late phone calls discussing research.
For his constant encouragement and his friendship.
\textbf{Babis Papamanthou}, for getting as excited about the append-only dictionary problem as me (if not more) and his constant efforts to help improve and position the result.

My collaborators at VMware Research Group (VRG).
\textbf{Dahlia Malkhi}, my first VRG mentor.
For all the fun-filled, three-hour meetings we had during the summer of 2017.
For bringing an incredibly-positive, highly-contagious energy to every meeting.
For her invaluable guidance.
For an incredibly productive summer that re-instilled confidence in me.
\textbf{Ittai Abraham}, my second VRG mentor, for our wonderful remote collaboration.
For all those long phone calls about blind signatures, threshold signatures and other fantastic beasts.
\textbf{Mike Reiter}\footnote{The author shall not acknowledge Mike Reiter's controversial victory in the VRG Chin-up Olympics of 2017, which was indubitably attributed to improper chin-up technique. Video evidence can be provided upon request.}, for his support when I needed it.
The only thing better than a meeting with Dahlia was a meeting with Mike and Dahlia.
\textbf{Benny Pinkas}, for all the helpful discussions about RSA threshold signatures, polynomial interpolation and Fast Fourier Transforms.
\textbf{Guy Golan-Gueta}, for always being open to me changing his existing PBFT codebase.
For all the long talks on speeding up threshold signatures in SBFT.
\textbf{Soumya Basu} and \textbf{Adi Seredinschi}, for all the productive conversations while at VMware and all the fun times while outside it.

All of my fellow interns from VRG, with whom I spent two fantastic summers in California.

The many folks that have helped and steered me along the way.
\textbf{Marten van Dijk}, for all the interesting discussions inside and outside our group meetings.
\textbf{Neha Narula}, for all the discussions about proof-of-work consensus.
\textbf{Madars Virza}, for our conversations on using polynomials to solve the append-only dictionary problem.
\textbf{Henry-Corrigan Gibbs}, for the conversation about accumulators and append-only dictionaries.
\textbf{Nickolai Zeldovich}, for letting me pick his brain on append-only dictionaries and for his advice on postdocs.
\textbf{Michael Alan Specter}, for all our conversations on applied crypto research.

\textbf{Nancy Lynch}, my academic advisor, whom I would meet with every semester to discuss my progress (or lack thereof).
I was always a bit nervous meeting with her ``empty handed'' in my first years.
Yet she always emanated a silent, reassuring confidence that I would find my way.

My family, for a loving home.

\textbf{My mother}, for all her sacrifices.
For raising us.
For letting us go.
For never doubting us.
For teaching us independence and responsibility through her faith in us.
For being a role model.
For her boundless love.

\textbf{My brother}, for crawling, walking and running through this life with me.
For letting me spend more time than him on the computer.

\textbf{My sister}, for loving us the moment she met us.
For being an inspiration (though she would probably laugh at this notion).

\textbf{My father}.
For his existence.
For giving me the rare opportunity to study in the United States.
For all his encouragement.

\textbf{My step-mother}.
For moving to Romania.
For teaching me how to write.
For taking care of my father.

\textbf{Nașu}, for being a father figure.

\textbf{Nașa}, for letting me keeping the TV on max volume so I could hear the English being spoken.
For her brilliant sense of humor.

\textbf{Mela}, for teaching me English.
For her faith in us.
For her limitless love.

My friends from MIT, for keeping me sane.
\textit{Ariel Anders}, \textit{Alex Băcanu}, \textit{Twan Koolen}, \textit{Marek Hempel}, \textit{Danielle Pace}, \textit{``Bald'' Mike}, \textit{Colm O'Rourke}, \textit{Gautam Kamath}, \textit{Sam Park}, \textit{Julius Adebayo}, \textit{Eva Golos}, \textit{Sara Achour}, \textit{Rohit Singh}, \textit{Michel ``Mon Amour...'' Babany}, \textit{Frank Permenter}, \textit{Orhan Celiker}, \textit{Jessica Ray}, \textit{Ognyan Georgiev}, \textit{Andre Urgiles}, \textit{Fernando Couto}, \textit{Cesar Cruz}, \textit{Veronica Mirta}, \textit{Avelino Tolentino}, \textit{Affi Maragh}, \textit{Ina Kundu}, \textit{Po-An (Ben) Tsai}, \textit{Chetan Manikantan}, \textit{David Qiu}, \textit{Garrett Dowdy}, \textit{João Ramos}, \textit{Lindsay Brownell}, \textit{Mark Jeffrey}, \textit{Pavel Chykov}.

The Romanian Mafia, at MIT and beyond, \textit{Sorin Grama}, \textit{Florin Chelaru}, \textit{Radu Berinde}, \textit{Răzvan ``Tati!'' Marinescu}, \textit{Iulia Mădălina Ștreangă}, \textit{Suzana Iacob}, \textit{Liliana Onița-Lenco}, \textit{Elena Solomon}, \textit{Andreea Bobu}, \textit{Andreea Bodnari}, \textit{Dana Jinaru}, \textit{Daniel Farmache}, \textit{Livia Dinu}, \textit{Ovidiu Tisler}, \textit{Teo Cucu}, \textit{Victor Pankratius}, \textit{Alin Dragoș}.

My gym bros, \textit{Sam ``Teach me how to deadlift'' Park}, \textit{Ivan ``It's spinal!'' Kuraj}, \textit{James Noraky}, \textit{Jun Wan} and \textit{Colm O'Rourke}, for all the fun times at the Z Center and Stata Center gyms.

My friends from college, \textit{Sharif Syed}, \textit{Brian Wilson}, \textit{Prashant Makwana}, \textit{Ezra Margono}, \textit{Mihai Albu}, \textit{Radu Zamfir}, \textit{Mike Boruta}, \textit{Luke Mladek}, \textit{Jan Kasiak}, \textit{Farhad Zaman}.

My friends from Romania, \textit{Alex ``Titisan'' Tiutiu}, \textit{Ana (Anița) Maria}, \textit{Ioana Bălan}, \textit{Maria Popescu}, \textit{Badea ``Mutu'' Mihai}, \textit{Cătălin ``Cole'' Apostol}, \textit{Liviu ``roacăru'' Nicolae}, \textit{Cristi ``Titus'' Niță}, \textit{Silviu ``But'' Niculae}, \textit{George ``Hesus'' Postelnicu}, \textit{Dana Velea}, \textit{Adrian Plăcintescu}, \textit{Bobi ``Lion Heart'' Niță}.

My physical therapist, \textbf{Dr. Lucia Hamilton}, who helped me fix my scapular dysfunction issue.
For putting up with me as her patient.
For teaching me not to over-think movement.
For the gift of ``lifting things up and putting them back down again''.

To everyone who has ever been a part of my life, thank you.
\cleardoublepage
\fi

%%%%%%%%%%%%%%%%%%%%%%%%%%%%%%%%%%%%%%%%%%%%%%%%%%%%%%%%%%%%%%%%%%%%%%
% -*-latex-*-
