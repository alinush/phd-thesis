A \textit{cryptographic accumulator}~\cite{acc-rsa,Nguyen05} is a commitment to a set of \textit{elements} $T = \{e_1, e_2, \dots, e_n\}$ referred to as the \textit{accumulated set}.
This commitment is often referred to as the \textit{accumulator}.
Accumulators have three key features.
First, \textit{membership} and \textit{non-membership} of any element can be proven w.r.t. the accumulator.
These proofs are called \textit{(non)membership witnesses} and are typically constant-sized.
Accumulators that support both membership and non-membership witnesses are called \textit{universal}~\cite{LLX07}.
Second, given two sets $T_1 \subseteq T_2$ and their accumulators $a_1$ and $a_2$, a \textit{subset witness} can be computed that convinces anyone with $a_1$ and $a_2$ that $T_1 \subseteq T_2$.
Third, a \textit{disjointness witness} can be computed for sets $T_1 \cap T_2 = \varnothing$ w.r.t. their accumulators.

A \textit{Merkle hash tree} over $n$ elements can be considered to be a cryptographic accumulator since, depending on how it is organized, it can offer some of the accumulator functionality (but not all).
In this thesis, however, we are concerned with accumulators that offer \textit{all} of the features from above.
This limits us to \textit{RSA accumulators} (see \cref{s:prelim:rsa-acc}), which we enhance in \cref{s:aas:from-rsa-acc:rsa-acc-enhance}, and \textit{bilinear accumulators} (see \cref{s:prelim:bilinear-acc}).