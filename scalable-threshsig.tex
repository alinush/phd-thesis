In this section, we show how to reduce the time to aggregate a $(t,n)$ BLS threshold signature from $\Theta(t^2)$ to $\Theta(t\log^2{t})$.
%We focus on BLS here, but our techniques generalize to other schemes.
Although we focus on BLS, our techniques can be used in any threshold cryptosystem (not just signatures) whose secret key lies in a prime-order field $\Fp$.
% Note: Must have roots of unity? No, can use other multiplication / division algorithms, but in practice, roots of unity are preferred.
This includes ElGamal signatures~\cite{Harn94,PK96,GJKR96}, ElGamal encryption~\cite{DF90}, Schnorr signatures~\cite{SS01,GJKR03} (but not RSA-based schemes, whose secret key does not lie in a prime-order field~\cite{Shoup00}).

Recall from \cref{s:prelim:threshsig} that $(t,n)$ threshold signature aggregation has two phases: (1) computing Lagrange coefficients and (2) exponentiating signature shares by these coefficients.
Unfortunately, when implemented naively in $\Theta(t^2)$ time, the time to compute Lagrange coefficients dominates the cost of aggregation (see \cref{f:thresh}).
In fact, current descriptions and implementations of threshold schemes all seem to use this inefficient scheme, which we dub \textit{naive Lagrange}~\cite{bls-chia-impl,bls-dfinity-impl,bls-herumi-impl,bls-sbft-impl,Boldyreva03}.
%We improve on naive Lagrange using known techniques for \textit{fast polynomial interpolation}~\cite{vG13ModernCh10} that, to the best of our knowledge, have not been applied to threshold cryptosystems before.

We make three contributions.
First, we adapt the \textit{fast polynomial interpolation} from \cite{vG13ModernCh10} to compute just the Lagrange coefficients fast in $\Theta(t\log^2{t})$ time.
We call this scheme \textit{fast Lagrange}.
Second, we speed up this scheme by using roots-of-unity rather than $\{1,2,\dots, n\}$ as the signer IDs.
Third, we implement threshold BLS signatures based on fast Lagrange and show they outperform the naive ones as early as $n=\blsOutperformN$ (see \cref{s:eval:threshsig}).

\subsection{Fast Lagrange-based BLS}
\label{s:threshsig:fast-lagr}
Recall that a \textit{Lagrange polynomial} $\Ell_i^T(x)$ is defined as:
\begin{align}
\Ell_i^T(x) &= \prod_{\substack{j\in T\\j\ne i}}\frac{x-j}{i-j}
\end{align}

First, let us rewrite $\Ell_i^T(x)$ as:
\begin{align}
\Ell_i^T(x) &= \frac{N_i(x)}{D_i}
\label{eq:lagr:ni}
\end{align}
\begin{align}
N_i(x) &= \frac{N(x)}{x-i} = \prod_{\substack{j\in T\\j\ne i}} {(x-j)}
\end{align}
\begin{align}
N(x) &= \prod_{i\in T}{(x-i)}
\end{align}
\begin{align}
D_i &= N_i(i) = \prod_{\substack{j\in T\\j\ne i}} {(i - j)}
\end{align}
Our goal is to quickly compute $\Ell_i^T(0)$ for each signer ID $i\in T$.
In other words, we need to quickly compute all $N_i(0)$'s and all $D_i$'s.

First, given the set of signer IDs $T$, we interpolate $N(x)$ in $\Theta(t\log^2{t})$ time by starting with the $(x-i)$'s as leaves of a tree and ``multiplying up the tree.''
Second, we can compute all $N_i(0) = N(0) / (-i)$ in $\Theta(t)$ time.
(Note that $N(0)$ is just the first coefficient of $N(x)$.)
However, computing $D_i,\forall i\in T$ appears to require $\Theta(t^2)$ time.
Fortunately, the \textit{derivative} $N'(x)$ of $N(x)$ evaluated at $i$ is exactly equal to $D_i$~\cite{vG13ModernCh10}.
Thus, an $\Theta(t\log^2{t})$ multipoint evaluation of $N'(x)$ at all $i\in T$ can efficiently compute all $D_i$'s!

To see why $N'(i) = D_i$, it is useful to look at the closed form formula for $N'(x)$ obtained by applying the product rule of differentiation (i.e., $(fg)' = f'g + fg'$).
For example, for $N(x) = (x-1)(x-2)(x-3)$:
\begin{align*}
N'(x)
%&= \big[(x-1)(x-2)\big]'(x-3) + \big[(x-1)(x-2)\big](x-3)'\\
%&= \big[(x-1)'(x-2) + (x-1)(x-2)'\big](x-3) + (x-1)(x-2)\\
%&= \big[(x-2)+(x-1)\big](x-3) + (x-1)(x-2)\\
&= (x-2)(x-3) + (x-1)(x-3) + (x-1)(x-2)\\
&= N_1(x) + N_2(x) + N_3(x)
\end{align*}
In general, we can prove that $N'(x) = \sum_{i\in T} N_i(x)\ \text{where}\ \deg{N'(\cdot)} = t-1$.
Since $N_j(i) = 0$ for all $i\ne j$ (see \cref{eq:lagr:ni}), it follows that $N'(i) = N_i(i) + 0 = D_i$.
Lastly, computing $N'(x)$ only takes $\Theta(t)$ time via polynomial differentiation.
Specifically, if $N = (c_t, c_{t-1}, \dots, c_1, c_0)$, then $N' = (t \cdot c_t, (t-1) c_{t-1}, \dots, 2 c_2, c_1)$.

To summarize, given a set $T$ of signer IDs, we can compute the Lagrange coefficients $\Ell_i^T(0) = N_i(x) / N'(i)$ by (1) computing $N(x)$ in $\Theta(t\log^2{t})$ time, (2) computing all $N_i(0)$'s in $\Theta(t)$ time, (3) computing $N'(x)$ in $\Theta(t)$ time and (4) evaluating $N'(x)$ at all $i\in T$ in $\Theta(t\log^2{t})$ time.
This reduces the time to compute all $\Ell_i^T(0)$'s from $\Theta(t^2)$ to $\Theta(t\log^2{t})$.

\subsection{Further Speed-ups via Roots of Unity}
\label{s:threshsig:roots-of-unity}
The fast Lagrange technique works for any threshold cryptosytem whose secret key $s$ lies in prime-order field $\Fp$.
However, for fields that support roots of unity, further speed-ups are possible.
(A caveat is that pairings on the underlying elliptic curve can be up to 2$\times$ slower.)
Without loss of generality, assume the total number of signers $n$ is a power of two and let $\omega_n$ denote a primitive $n$th root of unity in $\Fp$.
If we replace the $\{1,\dots,n\}$ signer IDs with roots of unity $\{\omega_n^{i-1}\}_{i\in[n]}$, then $N'(x)$ can be evaluated at any subset of signer IDs with a single Fast Fourier Transform (FFT).
This is much faster than a polynomial multipoint evaluation, which performs many polynomial divisions, each involving many FFTs.
Our fast Lagrange implementation from \cref{s:eval:threshsig} takes advantage of this optimization.
Furthermore, we use roots of unity to compute inverses faster in both our naive and fast Lagrange implementations (see \cref{s:eval:threshsig}).
For example, in naive Lagrange, we compute $N(0)=\prod_{i\in T}(0-\omega_n^i)$ much faster as $(-1)^{|T|}\cdot \omega_n^{\sum_{i\in T} i}$.
% Note: For both, we compute the numerators N(0)/(0-x_i) using the nice properties of roots of unity for -w_n^k and for inverses
