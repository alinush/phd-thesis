\newcommand{\rsaPrecompFigure}{
\begin{figure*}[t]
%\begin{sidewaysfigure}
{
    %\tiny
    %\scriptsize
    %\footnotesize
    %\small
    \normalsize
    %\large
    \begin{center}
        \begin{forest}
        for tree={
        %    fit=band,% spaces the tree out a little to avoid collisions
        %    fit=tight,% spaces the tree out less
        %    fit=rectangle,
            inner sep=4,
        }
        [{$g, \{e_1,\dots,e_8\}$}
            [{$g^{e_5 \cdots e_8}, \{e_1,\dots,e_4\}$}
                [{$g^{e_3 \cdots e_8}, \{e_1,e_2\}$}
                    [{$g^{e_2 \cdots e_8}, \{e_1\}$}
                        [, no edge, tier=odd ]
                    ]
                    [{$g^{e_1 e_3 \cdots e_8}, \{e_2\}$}
                        , tier=odd
                    ]
                ]
                [{$g^{e_1 e_2 e_5 \cdots e_8}, \{e_3,e_4\}$}
                    [{$g^{e_1 e_2 e_4 \cdots e_8}, \{e_3\}$}
                        [, no edge, tier=odd ]
                    ]
                    [{$g^{e_1 \cdots e_3, e_5 \cdots e_8}, \{e_4\}$}
                        , tier=odd
                    ]
                ]
            ]
            [{$g^{e_1 \cdots e_4}, \{e_5,\dots,e_8\}$}
                [{$g^{e_1 \cdots e_4 e_7 e_8}, \{e_5,e_6\}$}
                    [{$g^{e_1 \cdots e_4 e_6 \cdots e_8}, \{e_5\}$}
                        [, no edge, tier=odd ]
                    ]
                    [{$g^{e_1 \cdots e_5 e_7 e_8}, \{e_6\}$}
                        , tier=odd
                    ]
                ]
                [{$g^{e_1 \cdots e_6}, \{e_7,e_8\}$}
                    [{$g^{e_1 \cdots e_6 e_8}, \{e_7\}$}
                        [, no edge, tier=odd ]
                    ]
                    [{$g^{e_1 \cdots e_7}, \{e_8\}$}
                        , tier=odd
                    ]
                ]
            ]
        ]
        \end{forest}
    \end{center}
}
\caption{Divide-and-conquer approach by Sander et al.~\cite{SSY01} for precomputing $n$ membership witnesses for $T=\{e_1,\dots,e_n\}$ w.r.t. the RSA accumulator $\mathsf{acc}(T)$ in $\Theta(n\log{n})$ time. Here $n=8$.}
\label{f:rsa-acc-membproof-precomp}
\end{figure*}
%\end{sidewaysfigure}
}

Benaloh and de Mare~\cite{acc-rsa} were the first to introduce the notion of cryptographic accumulators and to instantiate it as \textit{RSA accumulators} via the Strong RSA assumption~\cite{BP97}.
Bari\'{c} and Pfitzmann~\cite{BP97} showed RSA accumulators are not just one-way functions but also collision-resistant.
Sander et al.~\cite{SSY01} showed how to efficiently precompute all witnesses in RSA accumulators in quasilinear time.
Li et al.~\cite{LLX07} extended RSA accumulators with non-membership witnesses.
Boneh et al.~\cite{BBF19} added support for computing and verifying witnesses in batch, for aggregating multiple witnesses into one and for union witnesses.

RSA accumulators have several advantages over bilinear accumulators.
First, RSA accumulators have constant-sized parameters, whereas bilinear accumulators need $q$-SDH public parameters (see \cref{def:q-sdh}) for committing to sets of size $\le q$.
Second, RSA accumulators do not require trusted setup if instantiated over \textit{class groups}~\cite{BBF19,Lipmaa2012} rather than the integers modulo $N$ where $N=pq$ is the product of two, unknown primes.
Third, RSA accumulators support precomputing all constant-sized membership witnesses in quasilinear time, which we discuss below.

RSA accumulators also have a few disadvantages.
First, committing to a set of size $n$ involves performing a big, $\lambda n$-bit sized exponentiation, which is inherently difficult to speed up via multi-threading.
(Indeed, this difficulty is leveraged to implement verifiable delay functions (VDFs)~\cite{Wesolowski19,Pietrzak2018,Boneh2018ASurvey}.)
Second, membership witnesses are much larger than in bilinear accumulators (e.g., 256 bytes versus 32 bytes when $|N|=2048$ bits).
Third, they do not support subset and disjointness witnesses, which we introduce in this thesis in \cref{s:aas:from-rsa-acc:rsa-acc-enhance}.

\paragraph{Setup.}
Unlike bilinear accumulators (see \cref{s:prelim:bilinear-acc}) which work over groups of prime (known) order, RSA accumulators work over \textit{hidden-order groups} such as the subgroup of quadratic residues $\QRN$ of $\ZNstar$ where $N=pq$ is the product of two primes.
We assume such a hidden-order group $\GroupHidOrd$ with generator $g$ has been set up.
For example, if this group is $\QRN$, we assume nobody knows the factorization of $N$.
(In \cref{s:prelim:trusted-setup-rsa}, we discuss how this can be achieved.)

\paragraph{Committing to a Set.}
Let $T = \{e_1, e_2, \dots, e_n\}$ denote a set of elements.
For RSA accumulators, the elements $e_i$ must be \textit{prime numbers}.
In particular, we will restrict ourselves to elements that are $2\lambda$-bit prime numbers.
A collision-resistant hash function (CRHF) $\Hprime$ can be used to map any elements from any domain to such $2\lambda$-bit primes.
To commit to $T$, one simply computes:
$$\mathsf{acc}(T) = g^{\prod_{e_i\in T} e_i}$$

\paragraph{Membership Witnesses.}
A membership witness for $e_i$ is just an RSA accumulator over $T\setminus\{e_i\}$:
$$\pi=\mathsf{acc}(T\setminus\{e_i\}) = g^{\prod_{e_j\in T\setminus\{e_i\}} e_j} = \mathsf{acc}(T)^{1/e_i}$$
Importantly, note that the witness $\pi$ cannot be computed efficiently by exponentiating $\mathsf{acc}(T)$ by $1/e_i$.
This is because $e_i^{-1}$ cannot be efficiently computed ``in the exponent'' of $\GroupHidOrd$, which is a hidden-order group.
Thus, the membership witness must be computed in a less efficient manner as:
$$g^{\prod_{e_j\in T\setminus\{e_i\}} e_j}$$
For this, the prover must do $n-1$ small, $2\lambda$-bit-sized exponentiations, which takes $\Theta(n)$ time.
(Alternatively, the prover can do one big, $\lambda (n-1)$-bit-sized exponentiation, which takes the same amount of time.)

To verify the membership witness for $e_i$, a verifier with $\mathsf{acc}(T)$ checks that:
\begin{align}
    \pi^{e_i} &= \mathsf{acc}(T) \Leftrightarrow\\
    \mathsf{acc}(T\setminus\{e_i\})^{e_i} &= \mathsf{acc}(T) \Leftrightarrow\\
    \left(\mathsf{acc}(T)^{1/e_i}\right)^{e_i} &= \mathsf{acc}(T) \Leftrightarrow\\
    \mathsf{acc}(T) &= \mathsf{acc}(T)
\end{align}
This way, membership witnesses can be verified in constant-time with a single exponentiation.
Membership witnesses can be proven secure under the Strong RSA assumption (see \cref{def:strong-rsa}).
RSA accumulators also support non-membership witnesses~\cite{LLX07}, but this thesis does not make (direct) use of them.

\paragraph{Membership Witness Aggregation.}
Boneh et al.~\cite{BBF19} show how to \textit{aggregate} two membership witnesses for two distinct elements into a single witness.
Specifically, given $\pi_1, \pi_2$ for elements $e_1$ and $e_2$ respectively, a single witness $\pi$ for the two elements can be obtained as follows.
First, \bezout coefficients $a,b$ such that $a e_1 + b e_2 = 1$ are calculated using the Extended Euclidean Algorithm (EEA).
Then, the aggregated witness is computed as $\pi=\pi_1^b \pi_2^a$.
To verify this aggregated witness, the verifier checks that $\pi^{e_1 e_2} = \mathsf{acc}(T)$.

To see why this works, note that $\pi$ is indeed equal to $\mathsf{acc}(T)^{\frac{1}{e_1 e_2}}$:
\begin{align*}
\pi &= \pi_1^b \pi_2^a\Leftrightarrow\\
              &= \left(\mathsf{acc}(T)^{1/e_1}\right)^b \left(\mathsf{acc}(T)^{1/e_2}\right)^a \Leftrightarrow\\
              &= \mathsf{acc}(T)^{b/e_1+a/e_2} \Leftrightarrow\\
              &= \mathsf{acc}(T)^{\frac{b e_2}{e_1 e_2}+\frac{a e_1}{e_1 e_2}} \Leftrightarrow\\
              &= \mathsf{acc}(T)^\frac{b e_2 + a e_1}{e_1 e_2} \Leftrightarrow\\
              &= \mathsf{acc}(T)^\frac{1}{e_1 e_2}
\end{align*}

Note that although aggregating proofs saves communication, the verification time remains as slow as verifying all the individual witnesses.
For example, in a hidden-order group, the $\pi^{e_1\cdot e_2}$ exponentiation is as slow as the individual $\pi_1^{e_1}$ and $\pi_2^{e_2}$ exponentiations since $e_1 \cdot e_2$ cannot be ``reduced'' in the exponent.
However, as pointed out in \cite{BBF19}, the verification time can be sped up using a proof of exponentiation (see \cref{s:prelim:poe}), although at the cost of a higher aggregation time.

\rsaPrecompFigure

\paragraph{Divide-and-Conquer Trick for Precomputing Membership Witnesses Fast.}
Even though computing a single membership witness takes $\Theta(n)$ time in an RSA accumulator, computing all $n$ membership witnesses can be done in $\Theta(n\log{n})$ time~\cite{SSY01}.
Without loss of generality, assume $n=2^m$ for some $m>0$.
The algorithm's starting input is the generator $g$ (i.e., the accumulator of the empty set) and the set $T$ of $n$ elements for which witnesses must be precomputed.
Let $T_L$ and $T_R$ of size $n/2$ each denote the left and right ``halves'' of $T$.
The algorithm computes one RSA accumulator $a_L$ over $T_L$ and another accumulator $a_R$ over $T_R$ in $\Theta(n)$ time.
Then, the algorithm calls itself recursively: first with $a_R$ and $T_L$ as input, and second with $a_L$ and $T_R$ as input.
As shown in \cref{f:rsa-acc-membproof-precomp}, this recursive computation produces a membership witness for each element $e_i$.

\paragraph{Subset and Disjointness Witnesses.}
Boneh et al.~\cite{BBF19} introduced new techniques for batching and aggregating (non)membership witnesses in RSA accumulators.
Although they introduce \textit{union witnesses}, which can be easily modified into a subset witness, they do not explicitly introduce disjointness witnesses.
In \cref{s:aas:from-rsa-acc:rsa-acc-enhance}, we show how to use their techniques to obtain disjointness witnesses and a more efficient subset witness.